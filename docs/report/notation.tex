%% Define bracket commands (normal, square and curly).
\newcommand{\brac} [1]  {\ensuremath{\left({#1}\right)}}
\newcommand{\sbrac}[1]  {\ensuremath{\left[{#1}\right]}}
\newcommand{\cbrac}[1]  {\ensuremath{\left\{{#1}\right\}}}
\newcommand{\abrac}[1]  {\ensuremath{\left\langle{#1}\right\rangle}}


%% Symbols

% General
\newcommand{\test}       {\ensuremath{^{*}}}
\newcommand{\testT}      {\ensuremath{^{*\top}\!}}
\newcommand{\ttest}      {\ensuremath{^{**}}}
\newcommand{\real}   [1] {\ensuremath{\mathbb{R}^{#1}}}
\newcommand{\natu}   [1] {\ensuremath{\mathbb{N}^{#1}}}
\newcommand{\ident}  [1] {\ensuremath{\mathbf{I}_{#1}}}

% Variables
\newcommand{\targs}     {\ensuremath{y}}
\newcommand{\targ}      {\ensuremath{\mathbf{y}}}
\newcommand{\targd}     {\ensuremath{\mathbf{y}\test}}
\newcommand{\targdT}    {\ensuremath{\mathbf{y}\testT}}
\newcommand{\ins}       {\ensuremath{\mathbf{X}}}
\newcommand{\inss}      {\ensuremath{\mathbf{x}}}
\newcommand{\locs}      {\ensuremath{\mathbf{z}}}
\newcommand{\minibatch} {\ensuremath{\mathbf{B}}}
\newcommand{\feat}      {\ensuremath{\boldsymbol\Phi}}
\newcommand{\feats}     {\ensuremath{\boldsymbol\phi}}
\newcommand{\featfunc}  {\ensuremath{\phi}}
\newcommand{\weights}   {\ensuremath{\mathbf{w}}}
\newcommand{\var}       {\ensuremath{\sigma^2}}
\newcommand{\pomean}     {\ensuremath{\mathbf{m}}}
\newcommand{\pocov}     {\ensuremath{\mathbf{C}}}
\newcommand{\jacob}[1]  {\ensuremath{\mathbf{J}_{#1}}}
\newcommand{\hess}[1]   {\ensuremath{\mathbf{H}_{#1}}}
\newcommand{\reg}       {\ensuremath{\lambda}}
\newcommand{\param}     {\ensuremath{\theta}}
\newcommand{\lparam}    {\ensuremath{\gamma}}
\newcommand{\hyper}     {\ensuremath{\alpha}}
\newcommand{\lrate}     {\ensuremath{\eta}}

% Gaussian Process
\newcommand{\kernl}     {\ensuremath{k}}
\newcommand{\Kernl}     {\ensuremath{\mathbf{k}}}
\newcommand{\KERNL}     {\ensuremath{\mathbf{K}}}


%% Operations
\newcommand{\T}          {\ensuremath{^{\!\top}}}
\newcommand{\inv}        {\ensuremath{^{\text{-}1}}}
\newcommand{\deter}[1]   {\ensuremath{\left|{#1}\right|}}
\newcommand{\trace}[1]   {\ensuremath{\text{tr}\!\brac{#1}}}
\newcommand{\diag}[1]    {\ensuremath{\text{diag}\!\brac{#1}}}
\newcommand{\expec}[2]   {\ensuremath{\abrac{#2}_{\!{#1}}}}
\newcommand{\expece}[2]  {\ensuremath{\mathbb{E}_{#1}\!\sbrac{#2}}}
\newcommand{\evar} [2]   {\ensuremath{\mathbb{V}_{#1}\!\sbrac{#2}}}
\newcommand{\KL}[2]      {\ensuremath{\text{KL}\!\sbrac{{#1}\!\parallel\!{#2}}}}
\newcommand{\entropy}[1] {\ensuremath{\mathbb{H}\sbrac{#1}}}
\newcommand{\lnorm}[2]   {\ensuremath{\left\|{#2}\right\|_{{#1}}}}


%% Functions, PDFs etc
\newcommand{\func} [3] {\ensuremath{{#1}_{#3}\!\brac{{#2}}}}
\newcommand{\ffunc} [2] {\func{f}{#1}{#2}}
\newcommand{\activ} [1] {\func{g}{#1}{}}
\newcommand{\Prob}  [1] {\ensuremath{P\!\brac{#1}}}
\newcommand{\ProbC} [2] {\ensuremath{P\!\left({#1}\middle\vert{#2}\right)}}
\newcommand{\iProb}  [1] {\ensuremath{P\inv\!\brac{#1}}}
\newcommand{\iProbC} [2] {\ensuremath{P\inv\!\left({#1}\middle\vert{#2}\right)}}
\newcommand{\quant}  [1] {\ensuremath{Q\!\brac{#1}}}
\newcommand{\prob}  [1] {\ensuremath{p\!\brac{#1}}}
\newcommand{\probC} [2] {\ensuremath{p\!\left({#1}\middle\vert{#2}\right)}}
\newcommand{\qrob}  [1] {\ensuremath{q\!\brac{#1}}}
\newcommand{\qrobC} [2] {\ensuremath{q\!\left({#1}\middle\vert{#2}\right)}}
\newcommand{\gaus}  [1] {\ensuremath{\mathcal{N}\!\brac{#1}}}
\newcommand{\gausC} [2] {\ensuremath{\mathcal{N}\!\left({#1}\middle\vert{#2}\right)}}
\newcommand{\bern}  [1] {\ensuremath{\textrm{Bern}\!\brac{#1}}}
\newcommand{\bernC} [2] {\ensuremath{\textrm{Bern}\!\left({#1}\middle\vert{#2}\right)}}
\newcommand{\kfunc} [2] {\ensuremath{\kernl\!\brac{{#1}, {#2}}}}
\newcommand{\expon} [2] {\ensuremath{{#1}\!\times\!10^{#2}}}
\newcommand{\bigo}  [1] {\ensuremath{\mathcal{O}\!\brac{{#1}}}}
\newcommand{\elbo}      {\ensuremath{\mathcal{L}}}


%% Operators
\DeclareMathOperator*{\argmax}{\operatorname*{argmax}}
\DeclareMathOperator*{\argmin}{\operatorname*{argmin}}
