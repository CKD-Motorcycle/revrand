\documentclass[11pt, oneside]{article}

% Packages
\usepackage{jmlr2e}
\usepackage{xcolor}
\usepackage[super]{nth}
\usepackage{mathtools, amsfonts}

% Package config
% \sectionfont{\normalfont\sffamily\bfseries}
% \subsectionfont{\normalfont\sffamily\bfseries}
\hypersetup{colorlinks,
    linkcolor={red!50!black},
    citecolor={blue!50!black},
    urlcolor={blue!80!black}
}

% Notation and macros
%% Define bracket commands (normal, square and curly).
\newcommand{\brac} [1]  {\ensuremath{\left({#1}\right)}}
\newcommand{\sbrac}[1]  {\ensuremath{\left[{#1}\right]}}
\newcommand{\cbrac}[1]  {\ensuremath{\left\{{#1}\right\}}}
\newcommand{\abrac}[1]  {\ensuremath{\left\langle{#1}\right\rangle}}


%% Symbols

% General
\newcommand{\test}       {\ensuremath{^{*}}}
\newcommand{\testT}      {\ensuremath{^{*\top}\!}}
\newcommand{\ttest}      {\ensuremath{^{**}}}
\newcommand{\real}   [1] {\ensuremath{\mathbb{R}^{#1}}}
\newcommand{\natu}   [1] {\ensuremath{\mathbb{N}^{#1}}}
\newcommand{\ident}  [1] {\ensuremath{\mathbf{I}_{#1}}}

% Variables
\newcommand{\targs}     {\ensuremath{y}}
\newcommand{\targ}      {\ensuremath{\mathbf{y}}}
\newcommand{\targd}     {\ensuremath{\mathbf{y}\test}}
\newcommand{\targdT}    {\ensuremath{\mathbf{y}\testT}}
\newcommand{\ins}       {\ensuremath{\mathbf{X}}}
\newcommand{\inss}      {\ensuremath{\mathbf{x}}}
\newcommand{\locs}      {\ensuremath{\mathbf{z}}}
\newcommand{\minibatch}  {\ensuremath{\mathbf{B}}}
\newcommand{\feats}     {\ensuremath{\boldsymbol\Phi}}
\newcommand{\featfunc}  {\ensuremath{\boldsymbol\phi}}
\newcommand{\weights}   {\ensuremath{\mathbf{w}}}
\newcommand{\cov}       {\ensuremath{\boldsymbol{\Sigma}}}
\newcommand{\pocov}     {\ensuremath{\mathbf{C}}}
\newcommand{\lapcov}    {\ensuremath{\hat{\mathbf{C}}}}
\newcommand{\mcov}      {\ensuremath{\mathbf{S}}}
\newcommand{\jacob}[1]  {\ensuremath{\mathbf{J}_{#1}}}
\newcommand{\hess}[1]   {\ensuremath{\mathbf{H}_{#1}}}
\newcommand{\reg}       {\ensuremath{\lambda}}
\newcommand{\param}     {\ensuremath{\theta}}
\newcommand{\hyper}     {\ensuremath{\alpha}}
\newcommand{\lrate}     {\ensuremath{\eta}}

% Gaussian Process
\newcommand{\kernl}     {\ensuremath{k}}
\newcommand{\Kernl}     {\ensuremath{\mathbf{k}}}
\newcommand{\KERNL}     {\ensuremath{\mathbf{K}}}


%% Operations
\newcommand{\T}          {\ensuremath{^{\!\top}}}
\newcommand{\inv}        {\ensuremath{^{\text{-}1}}}
\newcommand{\deter}[1]   {\ensuremath{\left|{#1}\right|}}
\newcommand{\trace}[1]   {\ensuremath{\text{tr}\!\brac{#1}}}
\newcommand{\diag}[1]    {\ensuremath{\text{diag}\!\brac{#1}}}
\newcommand{\expec}[2]   {\ensuremath{\abrac{#2}_{\!{#1}}}}
\newcommand{\expece}[2]  {\ensuremath{\mathbb{E}_{#1}\!\sbrac{#2}}}
\newcommand{\evar} [2]   {\ensuremath{\mathbb{V}_{#1}\!\sbrac{#2}}}
\newcommand{\KL}[2]      {\ensuremath{\text{KL}\!\sbrac{{#1}\!\parallel\!{#2}}}}
\newcommand{\entropy}[1] {\ensuremath{\mathbb{H}\sbrac{#1}}}
\newcommand{\lnorm}[2]   {\ensuremath{\left\|{#2}\right\|_{{#1}}}}


%% Functions, PDFs etc
\newcommand{\func}  [2] {\ensuremath{f_{#2}\!\brac{{#1}}}}
\newcommand{\ofunc} [3] {\ensuremath{{#1}_{#3}\!\brac{{#2}}}}
\newcommand{\Prob}  [1] {\ensuremath{P\!\brac{#1}}}
\newcommand{\ProbC} [2] {\ensuremath{P\!\left({#1}\middle\vert{#2}\right)}}
\newcommand{\iProb}  [1] {\ensuremath{P\inv\!\brac{#1}}}
\newcommand{\iProbC} [2] {\ensuremath{P\inv\!\left({#1}\middle\vert{#2}\right)}}
\newcommand{\quant}  [1] {\ensuremath{Q\!\brac{#1}}}
\newcommand{\prob}  [1] {\ensuremath{p\!\brac{#1}}}
\newcommand{\probC} [2] {\ensuremath{p\!\left({#1}\middle\vert{#2}\right)}}
\newcommand{\qrob}  [1] {\ensuremath{q\!\brac{#1}}}
\newcommand{\qrobC} [2] {\ensuremath{q\!\left({#1}\middle\vert{#2}\right)}}
\newcommand{\gaus}  [1] {\ensuremath{\mathcal{N}\!\brac{#1}}}
\newcommand{\gausC} [2] {\ensuremath{\mathcal{N}\!\left({#1}\middle\vert{#2}\right)}}
\newcommand{\bern}  [1] {\ensuremath{\textrm{Bern}\!\brac{#1}}}
\newcommand{\bernC} [2] {\ensuremath{\textrm{Bern}\!\left({#1}\middle\vert{#2}\right)}}
\newcommand{\kfunc} [2] {\ensuremath{\kernl\!\brac{{#1}, {#2}}}}
\newcommand{\expon} [2] {\ensuremath{{#1}\!\times\!10^{#2}}}
\newcommand{\bigo}  [1] {\ensuremath{\mathcal{O}\!\brac{{#1}}}}
\newcommand{\elbo}      {\ensuremath{\mathcal{L}}}


%% Operators
\DeclareMathOperator*{\argmax}{\operatorname*{argmax}}
\DeclareMathOperator*{\argmin}{\operatorname*{argmin}}


% Preamble

% Heading arguments are
% {volume}{year}{pages}{submitted}{published}{author-full-names}
% \jmlrheading{}{}{}{}{}{}

% Short headings should be running head and authors last names
\ShortHeadings{revrand Technical Report}{Steinberg, Tiao, Reid, McCalman and
    O'Callaghan}
\firstpageno{1}

\title{\emph{revrand}: Technical Report}
\author{\name Daniel Steinberg \email daniel.steinberg@data61.csiro.au \\
        \name Louis Tiao \email louis.tiao@data61.csiro.au \\
        \name Alistair Reid \email alistair.reid@data61.csiro.au \\
        \name Lachlan McCalman \email lachlan.mccalman@data61.csiro.au \\
        \name Simon O'Callaghan \email simon.ocallaghan@data61.csiro.au \\
        \addr DATA61, CSIRO \\
        Sydney, Australia}

\date{}

\begin{document}

\maketitle
% \vspace{-0.5cm}
% \noindent\makebox[\linewidth]{\rule{\linewidth}{0.8pt}}
% \vspace{0.3cm}

\begin{abstract}
    This is a technical report on the \emph{revrand} software library. This
    library implements various Bayesian linear models (Bayesian linear
    regression), approximate Gaussian processes and generalised linear models.
    These algorithms have been implemented such that they can be used for
    large-scale inference by using stochastic gradients. All of the algorithms
    in \emph{revrand} use a unified feature composition framework that allows
    for easy concatenation and selective application of regression basis
    functions.
\end{abstract}

\tableofcontents

\section{Core Algorithms}

\subsection{Stochastic Gradients and Variational Objective Functions}
\label{sub:stochvar}

Stochastic Gradients is now a ubiquitous method for optimisation when a whole
dataset does not fit in memory, or when optimisation has to be distributed
amongst many computational nodes.

When an objective function factorises over data,
\begin{equation}
    \ffunc{\ins, \param}{} = \sum_{\inss \in \ins} \ffunc{\inss, \param}{},
\end{equation}
a regular gradient descent would perform the following iterations to minimise
the function w.r.t.\ $\param$,
\begin{equation}
    \param_k := \param_{k-1} - \lrate_k \sum_{\inss \in \ins}
    \nabla_{\param} \ffunc{\inss_n, \param}{}\!|_{\param = \param_{k-1}},
\end{equation}
where $\lrate_k$ is the learning rate (step size) at iteration $k$. Stochastic
gradients proposes the following update,
\begin{equation}
    \param_k := \param_{k-1} - \lrate_k \sum_{\inss \in \minibatch}
    \nabla_{\param} \ffunc{\inss, \param}{}\!|_{\param = \param_{k-1}},
    \label{eq:sg}
\end{equation}
where $\minibatch \subset \ins$ is a mini-batch of the original dataset, where
$|\minibatch| \ll |\ins|$.

Unfortunately some objective functions to not entirely decompose over the data,
i.e. 
\begin{equation}
    \ffunc{\ins, \param}{} = \sum_{\inss \in \ins} \ffunc{\inss, \param}{}
    + \func{g}{\param}{}.
    \label{eq:objwconst}
\end{equation}
Let $M = |\minibatch|$ and $N = |\ins|$, then we divide the contribution of the
constant term amongst the mini-batches in stochastic gradients,
\begin{equation}
    \param_k := \param_{k-1} - \lrate_k \sum_{\inss \in \minibatch}
    \nabla_{\param} \ffunc{\inss, \param}{}\!|_{\param = \param_{k-1}}
    - \lrate_k \frac{M}{N} \nabla_{\param} 
    \func{g}{\param}{}\!|_{\param = \param_{k-1}}.
    \label{eq:wsg}
\end{equation} 
or, equivalently, boost the contribution of the mini-batch,
\begin{equation}
    \param_k := \param_{k-1} - \frac{N}{M} \lrate_k \sum_{\inss \in \minibatch}
    \nabla_{\param} \ffunc{\inss, \param}{}\!|_{\param = \param_{k-1}}
    - \lrate_k \nabla_{\param} 
    \func{g}{\param}{}\!|_{\param = \param_{k-1}}.
    \label{eq:wsg2}
\end{equation} 
This is particularly relevant for variational inference where the evidence
lower bound objective has a component independent of the data. For example, 
lets consider the model,
\begin{align}
    \text{Likelihood:} \quad &\prod^N_{n=1} \probC{\targs_n}{\param}, \\
    \text{prior:} \quad &\probC{\param}{\hyper},
\end{align}
where we want to learn the values of the hyper-parameters, $\hyper$. Minimising
negative log-marginal likelihood is a good objective in this instance, since we
don't care about the value(s) of $\param$,
\begin{equation}
    \argmin_\hyper - \log \int \prod^N_{n=1} \probC{\targs_n}{\param}
    \probC{\param}{\hyper} d \param.
    \label{eq:lml}
\end{equation}
There are two problems with this objective however, (1) it may not factor over
data and (2) the integral may be intractable, for instance, if the prior and
likelihood are not conjugate. In variational inference we use Jensen's
inequality to lower-bound log-marginal likelihood with a tractable objective
function called the evidence lower bound (ELBO),
\begin{align}
    \log \probC{\targ}{\hyper} =& \log \int 
        \prod^N_{n=1} \probC{\targs_n}{\param} 
        \probC{\param}{\hyper} d \param \nonumber \\
        =& \log \int 
        \frac{\prod_n \probC{\targs_n}{\param} \probC{\param}{\hyper}}
        {\qrob{\param}} \qrob{\param} d \param \nonumber \\
        \geq& \int \qrob{\param} \log \sbrac{%
            \frac{\prod_n \probC{\targs_n}{\param} 
            \probC{\param}{\hyper}}{\qrob{\param}}}
        d \param
\end{align}
where $\qrob{\param}$ is an approximation of $\probC{\param}{\hyper}$ that 
makes inference easier. This can be re-written as,
\begin{equation}
    \elbo = \sum^N_{n=1} \expec{q}{\log\probC{\targs_n}{\param}} -
    \KL{\qrob{\param}}{\probC{\param}{\hyper}},
    \label{eq:elbo}
\end{equation}
which takes the form of Equation~\eqref{eq:objwconst}, and so we can weight the
Kullback-Leibler term like the constant term, $\func{g}{\cdot}{}$, from
Equation~\eqref{eq:wsg} if we use stochastic gradients optimisation.
Furthermore, if $\qrob{\param} = \probC{\param}{\hyper}$ then the lower bound
is tight, and this will be equivalent to optimising log-marginal likelihood.


\subsection{Bayesian Linear Regression}

The first machine learning algorithm in revrand is a simple Bayesian linear
regressor of the following form,
\begin{align}
    \text{Likelihood:} \quad &\prod^N_{n=1} 
    \gausC{\targs_n}{\feats_n\T\weights, \var}, \\
    \text{prior:} \quad &\gausC{\weights}{\mathbf{0}, \reg\ident{D}},
\end{align}
where $\feats_n := \func{\featsym}{\inss_n, \param}{}$ is a feature, or basis,
function that $\featsym \real{d} \to \real{D}$. This is the same algorithm
described in~\citet[Chapter 2]{Rasmussen2006}. We then:
\begin{itemize}
    \item Optimise $\var, \reg$ and $\param$ w.r.t.\ log-marginal likelihood,
        \begin{equation}
            \log \probC{\targ}{\var, \reg, \param} =
            \log \gausC{\targ}{\mathbf{0}, \var\ident{N} + \reg \feat\T\feat},
        \end{equation}
        where $\feat \in \real{N \times D}$ is the concatenation of all the
        features, $\feats_n$. Note this results in the covariance of the
        log-marginal likelihood being $N \times N$, though we can use the
        Woodbury identity to simplify the corresponding matrix inversion.
    \item Solve analytically for the posterior over weights, $\weights | \targ
        \sim \gaus{\pomean, \pocov}$ given the above hyperparameters, where,
        \begin{align*}
            \pocov &= \sbrac{\reg \ident{D} + \frac{1}{\var}
                \feat\T \feat}\inv, \\
            \pomean &= \frac{1}{\var} \pocov \feat\T \targ.
        \end{align*}
    \item Use the predictive distribution
        \begin{align}
            \probC{\targs\test}{\targ, \ins, \inss\test} &= \int
            \gausC{\targs\test}{\feats\testT\weights, \var}
            \gausC{\weights}{\pomean, \pocov} d\weights, \nonumber \\
            &= \gausC{\targs\test}{\feats\testT \pomean,
                \var + \feats\testT \pocov \feats\test}
        \end{align}
        for query inputs, $\inss\test$. This gives us the useful expectations,
        \begin{align}
            \expece{}{\targs\test} &= \feats\testT\pomean, \\
            \evar{}{\targs\test} &= \var + \feats\testT\pocov\feats\testT.
        \end{align}
\end{itemize}

It is actually easier to use the ELBO form with stochastic gradients for
learning the parameters of this algorithm, rather than log-marginal likelihood
recast using the Woodbury identity.  This is because it is plainly in the same
form as Equation \eqref{eq:objwconst}, though it would give the same result as
log-marginal likelihood, the ``approximate'' posterior is the same form as the
true posterior, i.e.\ $\qrob{\weights} = \gausC{\weights}{\pomean, \pocov}$.
The ELBO for this model is,
\begin{equation}
    \elbo = \sum^N_{n=1} 
    \expec{q}{\log\gausC{\targs_n}{\feats_n\T\weights, \var}}
    - \KL{\gausC{\weights}{\pomean, \pocov}}
        {\gausC{\weights}{\mathbf{0}, \reg\ident{D}}}.
    \label{eq:slmobj}
\end{equation}
More specifically,
\begin{align*}
    \expec{q}{\log\gausC{\targs_n}{\feats_n\T\weights, \var}} =&
    \log \gausC{\targs_n}{\feats_n\T\pomean, \var}
    - \frac{1}{2 \var} \trace{\feats_n\T\feats_n\pocov}, \\
    \KL{\gausC{\weights}{\pomean, \pocov}}
        {\gausC{\weights}{\mathbf{0}, \reg\ident{D}}} =&
        \frac{1}{2 \reg} \sbrac{\trace{\pocov} + \pomean\T\pomean} 
        - \frac{1}{2} \log\deter{\pocov} \\
        &+ \frac{D}{2} \brac{\log \reg - 1}.
\end{align*}
We have not implemented a stochastic gradient version of this algorithm since
it still requires a determinant involving a $D \times D$ matrix, and so is
\bigo{D^3} in complexity, per iteration. This is true even if we optimise the
posterior covariance directly (or a triangular parameterisation). The GLM
presented in the next section circumvents this issue, and is more suited to
really large $N$ and $D$ problems.


\subsection{Bayesian Generalised Linear Models}

The algorithm of primary interest in \emph{revrand} is the Bayesian generalised
linear model. The general form of the model implemented by this algorithm is,
\begin{align}
    \text{Likelihood:} \quad &\prod^N_{n=1} 
        \probC{\targs_n}{\activ{\feats_n\T\weights}, \lparam}, 
        \label{eq:glmlike} \\
    \text{prior:} \quad &\gausC{\weights}{\mathbf{0}, \reg\ident{D}},
\end{align}
for an arbitrary univariate likelihood, $\prob{\cdot}$, with an appropriate
transformation (inverse link) function, $\activ{\cdot}$, and parameter(s),
$\lparam$. 

Naturally, both calculating the exact posterior over the weights,
$\probC{\weights}{\targ, \ins}$, and the log-marginal likelihood,
$\prob{\targ}$, for hyperparameter learning are intractable since we may have a
non-conjugate relationship between the likelihood and prior. Therefore we must
resort to approximating the true posterior and the log-marginal likelihood.

Firstly, we approximate the true posterior over weights with a mixture of $K$
diagonal Gaussians,
\begin{align}
    \probC{\weights}{\targ, \ins} &\approx \qrob{\weights}, \nonumber \\
    &= \frac{1}{K} \sum^K_{k=1} \gausC{\weights}{\pomean_k, \dpocov_k},
\end{align}
where $\dpocov_k = \diag{[\dpocovs_{k,1}, \ldots, \dpocovs_{k, D}]\T}$, which
is inspired from similar approximations made in \citet{gershman2012,
    nguyen2014automated}. This is a very flexible form for the approximate
posterior, and has the nice property that our algorithm no longer has a
\bigo{D^3} cost associated with the number of features.

Then we approximate the log marginal likelihood using auto-encoding variational
Bayes \citep{kingma2014auto}. The exact lower bound on log marginal likelihood
is,
\begin{equation}
    \elbo = \sum^N_{n=1} 
    \expec{q}{\log\probC{\targs_n}{\activ{\feats_n\T\weights}, \lparam}}
    - \KL{\textstyle\frac{1}{K}\textstyle\sum_{k}
            \gausC{\weights}{\pomean_k, \dpocov_k}}
        {\gausC{\weights}{\mathbf{0}, \reg\ident{D}}}.
    \label{eq:glmobj}
\end{equation}
This can be expanded,
\begin{multline}
    \elbo = \frac{1}{K} \sum^K_{k=1} \sum^N_{n=1} 
    \expec{q_k}{\log\probC{\targs_n}
        {\activ{\feats_n\T\weights}, \lparam}}
    + \frac{1}{K} \sum^K_{k=1}
        \expec{q_k}{\log\gausC{\weights}{\mathbf{0}, \reg\ident{D}}} \\
    + \entropy{\textstyle\frac{1}{K}\textstyle\sum_{k}
            \gausC{\weights}{\pomean_k, \dpocov_k}},
    \label{eq:glmobj_exp}
\end{multline}
but unfortunately there are two intractable integrals here, the expected log
likelihood, and the entropy of the Gaussian mixture. We can use the lower bound
on the entropy term also used in \citet{gershman2012, nguyen2014automated},
\begin{equation}
    \entropy{\textstyle\frac{1}{K}\textstyle\sum_{k}
        \gausC{\weights}{\pomean_k, \dpocov_k}} \geq
    - \frac{1}{K} \sum_{k=1}^K \log \sum_{j=1}^K \frac{1}{K}
    \gausC{\pomean_k}{\pomean_j, \dpocov_k + \dpocov_j}.
\end{equation}
We can then use the reparameterisation trick in auto-encoding variational Bayes
to sample the expected log likelihood,
\begin{equation}
    \sum^N_{n=1} 
    \expec{q_k}{\log\probC{\targs_n}
        {\activ{\feats_n\T\weights}, \lparam}} \approx
    \frac{1}{L} \sum^{L}_{l=1} \sum^N_{n=1} \log\probC{\targs_n}
    {\activ{\feats_n\T\func{f}{\pomean_k, \dpocov_k, \resamp^{(l)}}{k}},
        \lparam}
\end{equation}
where,
\begin{equation}
    \func{f}{\pomean_k, \dpocov_k, \resamp^{(l)}}{k} =
    \pomean_k + \sqrt{\dpocov_k} \odot \resamp^{(l)},
    \qquad \resamp^{(l)} \sim \gaus{\mathbf{0}, \ident{D}}.
\end{equation}
Here $\odot$ is the element-wise product. We can also use this trick to compute
approximate derivatives, $\frac{\partial}{\partial \alpha}
\expec{q(\alpha)}{\log \probC{\targs}{\alpha}} \approx \frac{1}{L} \sum^L_{l=1}
\frac{\partial}{\partial\alpha} \log \probC{\targs}{\func{f}{\alpha,
        \resamp^{(l)}}{}}$, which simplifies the implementation greatly! The
final auto-encoding variational Bayes objective for our GLM is,
\begin{multline}
    \elbo \approx \frac{1}{K} \sum^K_{k=1} \Bigg[
    \frac{1}{L} \sum^L_{l=1} \sum^N_{n=1} 
    \log\probC{\targs_n} {\activ{\feats_n\T f_k^{(l)}}, \lparam}
    + \log\gausC{\weights}{\mathbf{0}, \reg\ident{D}}
    - \frac{1}{2\reg} \trace{\dpocov_k} \\
    - \log \sum_{j=1}^K \frac{1}{K}
    \gausC{\pomean_k}{\pomean_j, \dpocov_k + \dpocov_j}
    \Bigg].
    \label{eq:glmobj_exp}
\end{multline}
We can straight forwardly use this objective with in a stochastic gradients
setting using with the tactic in Equations \eqref{eq:wsg} or \eqref{eq:wsg2}.

TODO: Gradients in appendix?

The most simple and accurate method for approximating the predictive
distribution, $\probC{\targs\test}{\targ, \ins, \inss\test}$ is to Monte-Carlo
sample the integral,
\begin{equation}
    \probC{\targs\test}{\targ, \ins, \inss\test} \approx
    \int \probC{\targ}{\activ{\feats\testT\weights}, \lparam}
    \frac{1}{K} \sum^K_{k=1} \gausC{\weights}{\pomean_k, \dpocov_k} d \weights.
    \label{eq:glppred}
\end{equation}
However, this integral is not particularly useful unless we wish to evaluate
known $\targ\test$ under the model. For prediction, it is more useful to
compute (using Monte-Carlo integration) the predictive expectation, 
\begin{align}
    \expece{}{\targs\test} \approx&
    \int \frac{1}{K} \sum^K_{k=1} \gausC{\weights}{\pomean_k, \dpocov_k}
    \int \targs\test \probC{\targs\test}{\activ{\feats\testT\weights}, \lparam}
    d \targs\test d \weights
    \nonumber\\
    =& \int \expece{}{\probC{\targs\test}
        {\activ{\feats\testT\weights}, \lparam}}
    \frac{1}{K} \sum^K_{k=1} \gausC{\weights}{\pomean_k, \dpocov_k}
    d \weights.
    \label{eq:glmexpec}
\end{align}
Often we find $\expece{}{\probC{\targs\test}{\activ{\feats\testT\weights},
        \lparam}} = \activ{\feats\testT\weights}$, however this is only true
with with right choice and usage of the activation function. Furthermore, it is
useful to compute quantiles of the predictive density in order to ascertain the
predictive uncertainty. We start by sampling the predictive cumulative density
function, $\Prob{\cdot}$,
\begin{align}
    &\ProbC{\targs\test \leq \cdfAlph}{\targ, \ins, \inss\test} \nonumber\\ 
    &\qquad\approx \int 
    \frac{1}{K} \sum^K_{k=1} \gausC{\weights}{\pomean_k, \dpocov_k}
    \int^{\cdfAlph}_{-\infty} 
    \probC{\targs\test}{\activ{\feats\testT\weights}, \lparam}
    d \targs\test d \weights \nonumber\\
    &\qquad= \int
    \ProbC{\targs\test \leq \cdfAlph}{\activ{\feats\testT\weights}, \lparam}
    \frac{1}{K} \sum^K_{k=1} \gausC{\weights}{\pomean_k, \dpocov_k}
    d \weights.
    \label{eq:lapexpec}
\end{align}
Once we have obtained sufficient samples from the (mixture) posterior we can
obtain quantiles, $\cdfAlph$, for some chosen level of probability, $p$, using
root finding techniques. Specifically, we use root finding techniques to solve
the following for $\cdfAlph$,
\begin{equation}
    \ProbC{\targs\test \leq \cdfAlph}{\targ, \ins, \inss\test} - p = 0.
\end{equation}

\subsection{Large Scale Gaussian Process Approximation}

% TODO: Show kernel approximations (from notebook)
% TODO: Some mention of these work with the SLM and GLM.

% Highlight how we can use proper optimisation to learn kernel parameters
% \emph{without} weight re-sampling (unlike what they say in ``The Mondrian
% Kernel'' paper). I.e. extend what Edwin wrote in the E/UKS paper in section 3 to
% show how we learn length-scales (this should be obvious from the A la Carte
% paper though).

TODO: re-write the following:

In \emph{revrand} we approximate Gaussian Processes with our standard and
generalised linear models by using random feature functions such as those of
\citeauthor{rahimi2007} \citeyearpar{rahimi2007,rahimi2008}. They use Bochner's
theorem regarding the relationship between a kernel and the Fourier transform
of a non-negative measure that (via Wiener-Khintchine's theorem) establishes
the duality of the covariance function of a stationary process and its spectral
density,
\begin{align}
	\kernl(\tfourier) &= \int \specfourier(\ffourier) 
    e^{i \ffourier\T  \tfourier} d \ffourier,  \\
	\specfourier(\ffourier) &= \int \kernl(\tfourier) 
    e^{- i \ffourier\T \tfourier}  d \tfourier.
	\label{eq:fourier}
\end{align}
\citeauthor{rahimi2007}'s  main insight \citeyearpar{rahimi2007} is that we can
approximate the kernel by constructing `suitable' random features and Monte 
Carlo averaging over samples from $\specfourier(\ffourier)$,
\begin{equation}
    \kernl(\inss - \inssprime) = \kernl(\tfourier) 
    \approx \frac{1}{D} \sum_{i=1}^{D} \singlefeatfunc{\inss}{i}\T\!
	\singlefeatfunc{\inssprime}{i},
	\label{eq:mcapprox}
\end{equation}
$\singlefeatfunc{\inss}{i}$ corresponds to the $i$th sample from the feature
map. An example of a feature vector construction in the above approximation is,
\begin{align}
	\nonumber
    \sbrac{\singlefeatfunc{\inss}{i}, \singlefeatfunc{\inss}{D+i}}&= 
    \frac{1}{\sqrt{D}} \sbrac{\cos(\ffourier_i^T \inss), 
    \sin(\ffourier_i^T \inss)}, \qquad \\
    \text{with}~\ffourier_i & \sim 
    \gausC{\ffourier_i}{\mathbf{0}, \varfeat \ident{d}},
\end{align}
for $i=1, \ldots, D$,  which in fact is a mapping into a $2 D$-dimensional
feature space.  \citet{rahimi2007} used the above feature map to approximate
the commonly used (isotropic) squared exponential kernel, and showed that such
an approximation converges in expectation to the true kernel.  

TODO: table of kernels and sampling distributions currently in revrand
(Laplace, Cauchy, RBF, Matern5/2, Matern3/2 etc)

TODO: FastFood

TODO: A la Carte spectral mixtures

\section{Experiments}

TODO: for now see the notebooks.

% \printbibliography%
\bibliography{report}

\end{document}
